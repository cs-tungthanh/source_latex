\chapter{GIỚI THIỆU ĐỀ TÀI}\label{chapter:introduction}
\pagestyle{fancy}

\section{Đặt vấn đề}
Ung thư là nguyên nhân gây tử vong hàng đầu tại các nước đang phát triển, đặc biệt tại Việt Nam. Hơn 95\% ung thư tế bào gan là các khối u ác tính.  Theo thống kê của WHO 2018, số ca mắc mới ung thư tại Việt Nam không ngừng tăng, từ 68.000 ca năm 2000 lên 126.000 năm 2010. Năm 2018, số ca ung thư mới tăng lên gần 165.000 ca/96,5 triệu dân, trong đó gần 70\% trường hợp tử vong, tương đương 115.000 ca. So sánh Việt Nam trên bản đồ ung thư thế giới, tỉ lệ mắc của Việt Nam không cao, tuy nhiên tỉ lệ tử vong tương đối lớn, xếp vị 56/185 quốc gia và vùng lãnh thổ với tỉ lệ 104,4/100.000 dân.\par

Ung thư là tên dùng chung để mô tả một nhóm các bệnh phản ảnh những sự thay đổi về sinh sản, tăng trưởng và chức năng của tế bào. Các tế bào bình thường trở nên bất thường và tăng sinh một cách không kiểm soát, xâm lấn các mô ở hay di căn qua hệ thống bạch huyết hay mạch máu. Di căn là nguyên nhân gây tử vong chính của ung thư. Ung thư có rất nhiều dạng, tùy theo từng vị trí phát tán bệnh trong cơ thể con người như: ung thư tuyến tiền liệt liệt, ung thư phổi, ung thư tụy,... đặc biệt là ung thư gan hay còn gọi là ung thư nguyên phát. Khi các tế bào ung thư gan hóa, gan không thể thực hiện được các chức năng thích hợp, dẫn đến các tác động có hại và nghiêm trong cho cơ thể. Bênh lí liên quan đến gan rất khó phát hiện ở giai đoạn đầu, hầu hết các triệu chứng sớm đều bị bỏ qua vì tưởng chừng chỉ là một phản ứng bình thường của cơ thể. Bên cạnh đó những tổn thương thường rất nhỏ, khó quan sát bằng phương pháp thông thường.\par 

Hiện nay, với sự phát triển của khoa học công nghệ các loại công cụ hỗ trợ cho việc chuẩn đoán ngày càng được cải thiện. Các kỹ thuật chụp ảnh y khoa ngày đang càng phát triển, điều đó mang lại cái nhìn tổng quan giúp các bác sĩ có thể chuẩn đoán được các tế bào tổn thương. Tuy nhiên việc phân tích kết quả từ cách ảnh thô của các công cụ này vẫn còn nhiều hạn chế bởi tính trực quan của chúng. Hiện nay đã có nhiều hệ thống cho phép phân tích ảnh y khoa một cách tự động nhưng độ chính xác đem lại chưa cao do đó sử dụng trong thực tiễn là điều không thể. Chính vì điều đó việc cung cấp một công cụ hỗ trợ bác sĩ trong quá trình chuẩn đoán bệnh là đề tài rất hứa hẹn trong thời điểm hiện nay.\par

Dựa trên tình hình phát triển của Cách mạng khoa học công nghệ 4.0, việc ứng dụng trí tuệ nhân tạo trong quá trình xử lý ảnh y khoa đang dần trở nên nóng hổi. Hiện nay nhiều nghiên cứu trong lĩnh vực này đã được tiến hành và mang lại những thành công đáng kinh ngạc. Bài toán xử lý ảnh trong lĩnh vực này trở nên khả thi với trình độ hiện tại của con người. Tuy lĩnh vực xử lý ảnh này chỉ mới nổi nhưng những thành tựu đạt được lại rất cao do đó nguồn tài liệu kham khảo hiện nay rất phong phú, điều đó khiến cho bài toán trở nên khả thi trong thời điểm hiện nay.\par

Do đó để hỗ trợ các bác sĩ trong công tác chẩn đoán và điều trị ung thư gan, chúng tôi đã lựa chọn đề tài ``Hệ thống làm nhãn cho ảnh chụp cắt lớp vi tính với sự trợ giúp của Trí tuệ nhân tạo''. Phân đoạn ảnh y khoa và trực quan hóa kết quả giúp quán trình chẩn đoán bệnh chính xác và nhanh chóng hơn, nhờ đó các bác sĩ sẽ có thêm thời gian để chăm sóc bệnh nhân. Nhận thấy số lượng dữ liệu y khoa để huấn luyện mạng học sâu còn hạn chế, nhóm đã thừa kế và phát triển hệ thống làm nhãn Data Annotation Tool để đáp ứng nhu cầu làm nhãn ảnh y khoa.\par

\section{Mục tiêu, phạm vi giới hạn của đề tài}
Để đảm bảo tính khả quan, phù hợp ở mức độ Luận văn tốt nghiệp, chúng tôi sẽ giới hạn lại phạm vi nguyên cứu cho lĩnh vực xử lý ảnh y khoa mà nhóm hướng tới. Ở đề tài này, nhóm chúng tôi sẽ tập trung phân tích với loại ảnh là ảnh chụp cắt lớp vi tính (3D CT) được chụp ở vùng bụng con người cụ thể hơn là bộ phận gan. Với công việc là phân đoạn 2 đối tượng chính: gan, mạch máu của gan bên cạnh đó kế thừa và phát triển công cụ làm nhãn của phòng thí nghiệm Graphics and Computer Vision Lab là phát triển một công cụ cho phép các chuyên gia y học có thể sử dụng để làm nhãn cho ảnh CT, từ đó giải quyết vấn đề thiếu dữ liệu cho công cuộc phát triển hiện nay. 

Mục tiêu cuối cùng là thiết kế một mô hình có khả năng kết hợp các kết quả phân đoạn các bộ phận của gan đạt độ chính xác cao nhất có thể. Từ đó hỗ trợ cho quá trình làm nhãn của các chuyên gia y khoa dễ dàng, nhanh chóng hơn. Ngoài ra việc trích xuất thành phần gan và mạch máu còn có khả năng hỗ trợ bác sĩ trong quá trình xác định các đặc tính quan trọng, khu trú của các thể bệnh thậm chí lập kế hoạch và hướng dẫn phẫu thuật một cách chính xác nhất. Xây dựng hệ thống để hỗ trợ thuận tiện cho các bác sĩ, người có chuyên môn trong lĩnh vực ảnh CT gắn nhãn cho ảnh, góp phần làm tăng tính đa dạng của dữ liệu. Tóm lại, hệ thống của chúng tôi tập trung vào các nhiệm vụ sau:
\begin{enumerate}[topsep=0pt,itemsep=0ex]
    \item Phân đoạn gan.
    \item Phân đoạn mạch máu của gan.
    \item Kế thừa và phát triển hệ thống làm nhãn ảnh y khoa.
\end{enumerate}

\section{Những thách thức}
Việc phân đoạn tự động gan và mạch máu đang phải đối mặt với những thách thức lớn. Đầu tiên, không có nhiều dữ liệu để đào tạo, đặc biệt dữ liệu có nhãn lại càng ít. Điều này là dễ hiểu bởi khó khăn trong việc thu thập dữ liệu cùng với việc gắn nhãn chính xác cần tới cấp độ chuyên gia. Bên cạnh đó, việc mất cân bằng dữ liệu trong tập dữ liệu là một thách thức đáng lưu ý, tỷ lệ xuất hiện của mạch máu rất nhỏ so với kích thước khung hình do đó gây ra trở ngại trong việc mất cân bằng giữa độ chính xác (precision) và độ truy hồi (recall). 

Thứ hai, hình ảnh của gan rất đa dạng, ở những người khác nhau và ở những lần chụp CT khác nhau sẽ đem lại kết quả khác nhau. Điều này xảy ra có thể do gan bị các cơ quan khác đè nén lên trong lúc chụp khiến cho ảnh CT của mỗi người vào mỗi thời điểm trở nên khác nhau.
\newpage
Ngoài ra, việc tính toán trên ảnh 3D yêu cầu sử dụng lượng tài nguyên lớn bởi kích thước ảnh của nó. Do đó để tối ưu được tốc độ tính toán hay khối lượng tài nguyên sử dụng là điều đáng lưu ý.

So với phân đoạn gan, phân đoạn mạch máu được xem là một nhiệm vụ khó khăn hơn không những bởi kích thước rất nhỏ bên cạnh đó hình dạng, vị trí xuất hiện của mạch máu là không theo một quy luật xác định nào. Tại những điểm phân chia mạch máu cũng dễ gây ra nhiều nhầm lẫn cũng gây cản trở lớn. Tiếp theo, do đặc trưng của ảnh CT, độ tương phản giữa gan và các cơ quan khác thường thấp. Trong các trường hợp có bệnh lý làm thay đổi cường độ HU của ảnh, việc phân đoạn các đối tượng càng khó khăn hơn. 
