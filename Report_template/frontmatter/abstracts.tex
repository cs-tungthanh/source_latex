\chapter*{Tóm tắt}
\addcontentsline{toc}{chapter}{Tóm tắt}
\onehalfspacing
\vspace{1.0cm}
Xử lý ảnh y khoa là một bài toán đã và đang được quan tâm trong nhiều năm gần đây bởi những lợi ích vượt bậc mà nó đem lại cho sự tiến bộ của ngành y học dựa trên sự phát triển của trí tuệ nhân tạo. Trong luận văn này, chúng tôi tiến hành nghiên cứu xây dựng hệ thống với mục tiêu hỗ trợ cho các bác sĩ trong việc chuẩn đoán các tổn thương trong gan một cách rõ ràng hơn, nâng cao độ chính xác so với phương pháp thủ công. Hệ thống được xây dựng dựa trên những công trình nghiên cứu, công nghệ chủ đạo dựa trên mô hình mạng học sâu (Deep Learning). \par

% Trong quá trình nghiên cứu, nhóm đã tiến hành tổng hợp, đánh giá ưu điểm, nhược điểm, cách thức hoạt động,... của các phương pháp mới nhất được đề xuất trong các hội nghị diễn ra gần đây. Bằng cách tiếp cận vấn đề theo nhiều hướng khác nhau, kế thừa và phát triển các nội dung của hệ thống có sẵn nhóm đã có được hướng nhìn tổng quát và chi tiết về nội dung đề tài thực hiện lấy cơ sở nền tảng cho việc đề xuất mô hình có khả năng cải thiện độ chính xác cao hơn, thiết thực hơn khi đưa ra hoạt động trên thực tiễn. \par

Trong luận văn này, chúng tôi nghiên cứu, xây dựng hệ thống tập trung vào ba nhiệm vụ chính: phân đoạn gan, phân đoạn mạch máu, cải thiện hệ thống làm nhãn cho ảnh chụp cắt lớp vi tính (CT). Chúng tôi tiến hành đề xuất mô hình U2net3D* dựa trên ý tưởng khối Residual U-Block được đề xuất trong \textit{bài toán phát hiện đối tượng nổi bật nhất (SOD)} cùng với mô hình tổng quát là mô hình mạng Unet được đề xuất trong cuộc thi \textit{ISBI challenge for segmentation of neuronal structures in electron microscopic stacks}. 

Về dữ liệu và độ đo, chúng tôi thực hiện đánh giá giữa mô hình cơ sở và mô hình chúng tôi đề xuất cho việc cải tiến trên 3 tập dữ liệu được công bố trong các cuộc thi: SLIVER07, LITS cho bộ phận gan và tập 3DIRCAD cho mạch máu của gan. Độ đo mà nhóm tập trung chính là hệ số tương đồng (Dice coefficient). 

Về thí nghiệm, chúng tôi thực hiện so sánh hiệu suất giữa mô hình cơ sở và mô hình chúng tôi đề xuất cho việc cải tiến, cùng với đó là các thí nghiệm giúp chúng tôi chọn các siêu tham số phù hợp với mô hình. 

Cuối cùng, chúng tôi trình bày các kết quả đạt được và định hướng phát triển trong tương lai đối với đề tài.

\textbf{Từ khóa:} liver vessel segmentation, liver, imbalancing loss, unet ..
\vskip 0.5cm
